\documentclass[conference]{IEEEtran}
\IEEEoverridecommandlockouts
% The preceding line is only needed to identify funding in the first footnote. If that is unneeded, please comment it out.
\usepackage{cite}
\usepackage{amsmath,amssymb,amsfonts}
\usepackage{algorithmic}
\usepackage{graphicx}
\usepackage{textcomp}
\usepackage{xcolor}
\def\BibTeX{{\rm B\kern-.05em{\sc i\kern-.025em b}\kern-.08em
    T\kern-.1667em\lower.7ex\hbox{E}\kern-.125emX}}
\begin{document}

\title{Distributed Mutual Exlcusion Algorithms: A Comparison of Central Server,
Ring Token and Multicast\\
{\footnotesize COMPSYS 725: Distributed Cyber-Physical Systems}
}

\author{\IEEEauthorblockN{Matt Eden}
\IEEEauthorblockA{\textit{Department of Electrical, Computer and Software
Engineering} \\
\textit{University of Auckland}\\
Auckland, New Zealand \\
mede607@aucklanduni.ac.nz}
}

\maketitle

%\begin{abstract}
%\end{abstract}

\begin{IEEEkeywords}
central server, ring token, multicast, mutual exclusion, distributed systems
\end{IEEEkeywords}

\section{Mutual Exclusion Algorithms}
\subsection{Overview}
Mutual exclusion is generally concerned with preventing interference and
ensuring consistency with resource access. For operating systems, this can be
managed with relative ease. However, distributed systems do not have the
shared variables or facilities that would be supplied by a single local kernel,
so a different approach is required. This is where distributed mutual exclusion
algorithms come into play, and a few of these are discussed in the following
subsections. \\
There are a few key assumptions made in this report that should be
acknowledged. The system being considered is comprised of \textit{N} processes
\textit{p\textsubscript{i}} with $i=1,2,3... N$. These process do not share
variables, but access common resources. These resources are contained within
a single critical section, and the system as a whole is asynchronous. The
failure of processes is not considered, with message delivery being reliable so
that any message sent is guaranteed to be delivered eventually and delivered
exactly once. No message loss or duplication occurs.
\subsection{Central Server}
\subsection{Ring Token}
\subsection{Multicast}

{\BibTeX} does not work by magic. It doesn't get the bibliographic
data from thin air but from .bib files. If you use {\BibTeX} to produce a
bibliography you must send the .bib files. 


\section{Comparison of Algorithms}
There are a few requirements for mutual exclusion, such as:
\begin{itemize}
  \item Safety
  \item Liveness
  \item Fairness
\end{itemize}
Of these, \textit{Safety} and \textit{Liveness} are considered essential
requirements.
Performance of any one algorithm is evaluated against a set of fixed criteria,
defined as follows.
\begin{itemize}
  \item Consumed Bandwidth
  \item Client Delay
  \item Effect on Throughput
\end{itemize}

\begin{table}[htbp]
\caption{Table Type Styles}
\begin{center}
\begin{tabular}{|c|c|c|c|}
\hline
\textbf{Table}&\multicolumn{3}{|c|}{\textbf{Table Column Head}} \\
\cline{2-4} 
\textbf{Algorithm} & \textbf{\textit{Bandiwidth}}& \textbf{\textit{Client
  Delay}}& \textbf{\textit{Throughput}} \\
\hline
Central Server & More table copy & &  \\
Ring Token & More table copy & &  \\
Multicast & More table copy & &  \\
\hline
\multicolumn{4}{l}{$^{\mathrm{a}}$Sample of a Table footnote.}
\end{tabular}
\label{tab1}
\end{center}
\end{table}

%\begin{figure}[htbp]
%\centerline{\includegraphics{fig1.png}}
%\caption{Example of a figure caption.}
%\label{fig}
%\end{figure}

\section*{Acknowledgement}
This report acknowledges the teachings of Dr. Avinash Malik and Ms. Jesin James
in the course COMPSYS 725: Distributed Cyber-Physical Systems taught at the
University of Auckland in Semester Two of the year 2020.

\begin{thebibliography}{00}
\bibitem{b1} George F. Coulouris, Jean Dollimore, Tim Kindberg and Gordon
  Blair, ``Distributed Systems: Concepts and Designs'', 5th ed, Boston,
    Massachusetts, Addison-Wesley; Pearson Education, 2011, pp. 41-49.
\end{thebibliography}
\vspace{12pt}
\end{document}
